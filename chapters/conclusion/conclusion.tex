\section{Conclusion}
L'exploitation du code source de génération des noms russes a permis de comprendre le fonctionnement des réseaux de neurones récurrents (RNN) et des cellules GRU (Gates Recurrent Unit) pour la génération de texte.\\
Ces résultats permettre de mieux choisir le bon modèle pour la génération des mots de passes, et d'optimiser les hyperparamètres pour obtenir des résultats satisfaisants.\\
Le tableau suivant résume les choix à faire pour le modèle de génération des mots de passes:

\begin{table}[H]
    \centering
    \begin{tabularx}{\textwidth}{|X|X|X|}
        \hline
        \textbf{Modèle}              & +++ & RNN est le modèle optimal pour ce problème. \\
        \hline
        \textbf{Taux d'aprentissage} & +++ & Une valeur autour de 0.0001 est idéale      \\
        \hline
        \textbf{Précision}           & +   & Une précision de 12\% (faible)              \\
        \hline
        \textbf{Époques}             & +++ & Inutile de dépasser 2,000 époques           \\
        \hline
        \textbf{Output}              & ++  & Génération des noms russes inexistantes     \\
        \hline
    \end{tabularx}
    \caption{Résumé des choix pour le modèle de génération des mots de passes}
    \label{tab:conclusion}
\end{table}