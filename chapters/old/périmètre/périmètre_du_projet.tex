\section{Périmètre du projet}
\subsection{Fonctionnalités attendues}
La solution proposée doit être capable de fournir plusieurs fonctionnalités aux utilisateurs tels que:
\begin{itemize}
    \item Temps d'entrainement minimal.
    \item Générer des mots de passes qui respectent les critères de sécurité.
    \item Utilisation simple et non complexe à l'aide d'une interface graphique facile à utiliser.
\end{itemize}
\subsection{Technologies utilisées}
Afin de pouvoir élaborer ce projet, on a eu recours aux différents languages de programmation, frameworks et outils techniques tels que:
\begin{itemize}
    \item \textbf{Python:} Pour le développement du réseau de neurones et d'un serveur qui permettra la communication entre une application et/ou un site web avec le modèle entrainé.
    \item \textbf{Flutter:} Pour le développement d'une application mobile/web qui sera utilisée par l'utilisateur pour utiliser le modèle de génération de mots de passes.
\end{itemize}