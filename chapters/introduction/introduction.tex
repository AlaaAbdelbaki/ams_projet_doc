\section{Introduction}

\subsection{Contexte}
Le projet \textbf{"Factory de mots de passes"} s'inscrit dans le contexte de l'apprentissage automatique pour la génération des mots de passes. Il consiste à développer une application qui implémente un réseau de neuronnes avec des options permettant d'entraîner le modèle sur des données d'entrée et l'évaluer en générant des nouvelles séquences à partir de quelques caractères de départ entrés par l'utilisateur.\\
Cette solution permet ainsi de générer automatiquement des mots de passes qui respectent les conventions d'une entreprise ou d'un domaine spécifique.
\subsection{Objectifs du projet}
Ce projet a les objectifs suivants à atteindre:
\begin{itemize}
    \item Développer un modèle intelligent en utilisant la bibliothèque PyTorch \cite{pytorch} de Python \cite{python}.
    \item Développer des fonctions d'entrainement et d'évaluation pour le modèle.
    \item Préparer un corpus d'entrainement contenant des mots de passes respectant une convention.
    \item Développer une solution logicielle pour mieux utiliser la solution intelligente.
\end{itemize}
\subsection{Portée}
Cette application sera destinée pour un grand public qui veut générer des mots de passes facile à mémoriser en entrant un mot clé.

